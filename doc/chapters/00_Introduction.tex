% Introduction (1-2 pages)
\chapter*{Увод}
Първоначално, когато компютрите започват масово да се разпространяват, те могат
да се достъпват единствено през терминал - текстови интерфейс, с който се 
работи с клавиатура. Не след дълго, поради бързия технологичен напредък, 
графичния потребителски интерфейс (ГПИ) бива създаден заедно с първата
компютърна мишка. Това развитие значително увеличава достъпността и лекотата на
използване на компютърните устройства. \\

Днес, около половин век по-късно, почти всеки притежава устройство с ГПИ.
Текстовият потребителски интерфейс (ТПИ) се използва доста по-рядко - предимно
в среди, където няма ГПИ, например при сървъри с ограничени ресурси или от
потребители с големи възможности. Поради тази причина повечето съществуващи
програми за ТПИ съществуват от десетилетия, а новите са принудени да използват 
софтуер, разработен преди десетки години. Това не намалява функционалността на
програмите, но увеличава времето им за разработка и може да ги направи
по-недостъпни за потребителя. Този проект решава именно този проблем. \\

% At first, after the invention of computers, users could
% only interact with them through terminals - a text
% interface requiring the use of a keyboard. Not long after,
% with the rapid technological advancement, the graphical
% user interface was first introduced alongside the computer
% mouse. It vastly increased the ease of use and
% accessability to the user and now, almost half a century
% later, nearly everyone owns a device with a graphical user
% interface. This advancement means that the text user 
% interface is used quite seldom - mainly in environments
% that don't support GUIs (for instance servers with limited 
% rescources) or by power users who seek an efficient 
% workflow. And so existing programms for the TUI are decades
% old and new ones are forced to use that same decade old
% technology. This doesn't impede functionality, but what it
% can do is increase development time and make the program
% harder to use. This thesis intends to solve exactly that
% problem.

За цел на проекта е поставена разработката на лесна за ползване библиотека,
която да позволява създаването на модерен потребителски интерфейс.

\section*{Задачи}
        \begin{itemize}
                \item Логическо отделени елементи на потребителския интерфейс.
                \item Оформление на компонентите на потребителския интерфейс.
                \item Обработка на събития от мишка и клавиатура.
        \end{itemize}
\vspace{10mm}

В \textbf{Глава 1} е направен обзор на средите на работа на ТПИ и методите за
управлението им. Също така е направен преглед на съществуващи библиотеки, които
се използват от приложения и разработчици по целия свят.

