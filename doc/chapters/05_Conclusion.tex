% conclusion (1 page)
\chapter{Заключение}

Настоящата дипломна работа реализира базов вариант на библиотека за създаване 
на текстов потребителски интерфейс. Библиотеката бе написана на Python, като
е достъпна за платформата Linux. Тя представлява абстрактен интерфейс, 
посредством който могат да се изразят и опишат логически компоненти, което
позволява изграждането на приложения с по-богата функционалност. Библиотеката
успешно разпознава и категоризира събития от клавиатура и мишка, предоставя
механизъм за стилизиране на елементи и оптимизира принтирането в терминал.

Възможностите за бъдещо развитие на проекта са многобройни. Едни от най-важните
функционалности са динамично уразмеряване и стилициране на компоненти. 
Други полезни наскои са поддръжка на цветове, скролбар за компоненти, 
припокриване на компоненти (3-то измерение), диагностични записи, поддръжка на 
повече терминални емулатори, markdown език, които описва компонентото дърво,
повече вградени компоненти и други.
